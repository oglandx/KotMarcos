\documentclass[11pt]{beamer}
\usetheme{Madrid}
\usepackage[utf8]{inputenc}
\usepackage[russian]{babel}
\usepackage{amsmath}
\usepackage{amsfonts}
\usepackage{amssymb}
\usepackage{here}
\beamertemplatenavigationsymbolsempty
\author[Косолапов С.А.]{Косолапов Семен Александрович}
\title[Промежуточная аттестация 2]{Разработка системы макросов для языка Kotlin}
\date{\today} 
\institute[СПбПУ]{
Руководитель: Беляев М.А.
}
\begin{document}


\begin{frame}
\titlepage
\end{frame}


\begin{frame}
\frametitle{Что требуется?}

Разработать препроцессор макроопределений, расширяющих язык Kotlin.

\begin{itemize}
	\item На входе - программа на расширенном языке Kotlin с конструкциями, позволяющими определять и применять макро
	\item На выходе - программа на <<чистом>> языке Kotlin
\end{itemize}

\end{frame}


\begin{frame}
\frametitle{Используемые принципы}

\begin{itemize}
	\item Macro-by-example
	\item Hygienic macros
\end{itemize}

\end{frame}


\begin{frame}
\frametitle{Что было сделано?}

\begin{itemize}
	\item Знакомство с кодом компилятора языка Kotlin
	\item Определён начальный вариант синтаксиса
	\item Начато написание кода
\end{itemize}

\end{frame}


\begin{frame}
\frametitle{Что предстоит сделать?}
\begin{itemize}
	\item Всё остальное
\end{itemize}	
\end{frame}


\begin{frame}
\frametitle{Что предстоит сделать?}
\begin{itemize}
	\item Программу, выполняющую простейшие преобразования
	\item Реализация принципа macro-by-example
	\item Обеспечить гигиену
	\item ???
\end{itemize}	
\end{frame}


\begin{frame}
\huge{\centerline{Спасибо за внимание!}}
\end{frame}


\end{document}
